\documentclass[letterpaper,10pt,fleqn,draftclsnofoot,onecolumn]{IEEEtran}
\usepackage{graphicx}                                        
\usepackage{amssymb}                                         
\usepackage{amsmath}                                         
\usepackage{amsthm}                                          
\usepackage{alltt}                                           
\usepackage{float}
\usepackage{color}
\usepackage{url}
\usepackage{balance}
\usepackage{enumitem}
\usepackage{geometry}
\geometry{textheight=8.5in, textwidth=6in}
\newcommand{\cred}[1]{{\color{red}#1}}
\newcommand{\cblue}[1]{{\color{blue}#1}}
\usepackage{hyperref}
\usepackage{geometry}
\usepackage{longtable,hyperref}
\newcommand{\longtableendfoot}


\title{CS444 Homework 2}
\def\name{Lilliana Watts, Nicholas Skinner, Yong Ping Li \\\today}
\author{\name}
\begin{document}
	\maketitle
	\hrulefill
	\section*{Design Plan}
	Placeholder
	
	\section*{Writeup Questions}
	
	\subsection{What do you think the main point of this assignment is?}
	$\bullet$ The main point of this assignment was to guide us into investigating, and learning about I/O schedulers, the Linux block layer, as well as how elevator algorithms operate. This assignment is also intended to help familiarize us with modifying some of the given QEMU command options.
	
	\subsection{How did you personally approach the problem? Design decisions, algorithm, etc.}
	%[1] and [2] are placeholders for bibliography
	$\bullet$ Placeholder
	
	
	\subsection{How did you ensure your solution was correct? Testing details, for instance.}
	$\bullet$ Placeholder
	
	\subsection{What did you learn?}
	$\bullet$ We learned about how the block layer, elevator algorithms, and the I/O schedulers operate to organize and carry out write requests on a drive. We also learned how we can disable the virtual I/O used on out previous QEMU command line.
	
	\subsection{How should the TA evaluate your work? Provide detailed steps to prove correctness.}
	$\bullet$ Placeholder
	 
	\section*{ Github Log }
	\begin{tabular}{l l p{1.5in}}\textbf{Detail} & \textbf{Author} & \textbf{Description}\\\hline
		\href{https://github.com/Skinnern/CS444/commit/9138e6857021329354027c49c35fea1604fd5585}{9138e68} & Nicholas & Placeholder\\\hline
	\end{tabular}
	%Table of Github commits
	
	%1 & 0 & 2018-04-03 & InitialCommit & 0 & 0 & 0 \\
%\end{longtable}
			%legend for table:
	%V & Tag & Date & Commit Message & ML & AL & DL
	%V 				- version ( just use the number after the commit above)
	%tag 			- in case you need to make 
	%Date 			- date it was pushed
	%Commit Message	- Commit message along the commit
	%ML 			- Modified lines
	%AL 			- Added lines
	%DL 			- Deleted lines
	
	\section*{Work Log}
	\begin{tabular}{l l c p{1.1in}}\textbf{Date} & \textbf{Name} & \textbf{Hours} & \textbf{Description}\\\hline
		4/30 & Nicholas & 1 & Setup Document\\\hline
		5/1 & Nicholas & 1 & Work on writeup\\\hline
	\end{tabular}
		
		\nocite{*}
		\bibliographystyle{plain}
		\bibliography{refs_a1}
	
\end{document}

